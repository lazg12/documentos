\chapter[Funciones]{Funciones}
\raggedright
Una función es un grupo de sentencias que realizan una tarea concreta y como resultado de ella retorna un único valor como respuesta, caso contrario si devuelve cero o más de un valor no es función.

\vspace{1em}
El retorno de la función se guarda en el nombre de esta, conceptualmente se puede decir o afirmar que el nombre de la función tiene asignado un lugar en la memoria, dicho de otra forma es parecida a como funciona una variable, pero en el ámbito estricto y riguroso no es variable.

\vspace{1em}
Al declarar una función el compilador de C, guarda un espacio de memoria para el nombre de la función.

\section[Declaración]{Declaración de funciones}
A continuación se muestra la sintaxis para declarar una función:
 
\begin{verbatim}
	tipoDeRetorno nombreDeLaFunción(listaDeParámetros){
		cuerpoDeLaFunción
		return (espresión);
	}
\end{verbatim}

A continuación se muestra una función que suma dos números y devuelve el resultado:

\begin{lstlisting}[language=C, caption={Función que suma dos números y devuelve el resultado}, style=codigoenc]
	int suma(int num1, int num2){
		int resultado;
		
		resultado = num1 + num2;
		return resultado;
	}
\end{lstlisting}	