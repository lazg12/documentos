\chapter[Estructuras de control]{Estructuras de control}
\raggedright
Permiten modificar el flujo del programa.

\vspace{1em}
Existen 3 estructuras de control básicas.

\vspace{1em}
\begin{itemize}
	\item \textbf{Secuencia}: Se ejecutan los comandos uno tras otro.
	\item \textbf{Selección}: También llamada decisión, y de acuerdo a la condición cambia el flujo del programa.
	\item \textbf{Repetición}: Es un bucle que ejecuta los comandos hasta que la condición sea falsa.
\end{itemize}

\section [Sentencias]{Sentencias}
Son los comandos que ejecuta el programa y pueden ser simples o compuestas.


\vspace{1em}
{\textbf{Sentencias simples}: Son las que finalizan con punto y coma, por ejemplo: \verb|printf("Hola Mundo");|}

\vspace{1em}
{\textbf{Sentencias compuestas}: Son el conjunto de sentencias simples encerradas entre llaves, ejemplo:}

\begin{verbatim}
	{
		sentencia1;
		sentencia2;
	}
\end{verbatim}	


\section [Selección if]{Selección if}
Es la principal sentencia de selección en C, ejemplo:

\begin{verbatim}
	if(condición)
		sentencia
\end{verbatim}

\section [Selección if - else]{Selección if - else}
Si no se cumple la condición cambia el flujo del programa, ejemplo:

\begin{verbatim}
	if(condición)
		sentencia
	else
		sentecia
\end{verbatim}

\section [Selección anidada]{Selección anidada}
Es cuando la estructura de control tipo selección tiene dentro del cuerpo una o más estructuras.

\section [Ejercicio]{Ejercicio}
Hacer un programa que ordene tres números.

\begin{lstlisting}[language=C, caption={Ordena tres números de forma ascendente}, style=codigoenc]
	#include <stdio.h>
	
	int main(void){
		int numero1;
		int numero2;
		int numero3;
		
		printf("====== PROGRAMA QUE ORDENA ASCENDENTEMENTE TRES N%cMEROS ======\n\n", 223);
		printf("Ingrese el primer n%cmero: ", 163);
		scanf("%d", &numero1);
		printf("\n");
		printf("Ingrese el segundo n%cmero: ", 163);
		scanf("%d", &numero2);
		printf("\n");
		printf("Ingrese el tercer n%cmero: ", 163);
		scanf("%d", &numero3);
		printf("\n");
		if((numero1 == numero2) && (numero1 == numero3)){
			printf("Los tres n%cmeros son iguales y su valor es: %d", 163, numero1);
		}else{
			if((numero1 < numero2) && (numero1 < numero3)){
				if(numero2 < numero3){
					printf("El orden es: %d, %d, %d", numero1, numero2, numero3);
				}else{
					printf("El orden es: %d, %d, %d", numero1, numero3, numero2);
				}
			}else{
				if(numero2 < numero3){
					if(numero1 < numero3){
						printf("El orden es: %d, %d, %d", numero2, numero1, numero3);	
					}else{
						printf("El orden es: %d, %d, %d", numero2, numero3, numero1);
					}
				}else{
					if(numero1 < numero2){
						printf("El orden es: %d, %d, %d", numero3, numero1, numero2);
					}else{
						printf("El orden es: %d, %d, %d", numero3, numero2, numero1);
					}	
				}
			}
		}
	}
\end{lstlisting}

\section [Selección switch]{Selección switch}

Se utilizan para seleccionar una de entre múltiples alternativas.

\vspace{1em}
Su selección se basa en el valor de una variable o de una expresión simple.

\vspace{1em}
La sintaxis se muestra a continuación:

\begin{verbatim}
	switch(selector){
		case etiqueta: sentencias; break;
		case etiqueta: sentencias; break;
		...
		default: sentencias; /* OPCIONAL*/
	}
\end{verbatim}

\vspace{1em}
{Tener en cuenta que el selector solo puede ser de tipo int o char.}

\section [Ejercicio]{Ejercicio}
Hacer un programa que muestre un mensaje dependiendo del tipo de nota, si es A sobresaliente, B Excelente, C muy bueno, D pasable, E malo, F muy malo.

\begin{lstlisting}[language=C, caption={Mostrar mensaje según la nota}, style=codigoenc]
	#include <stdio.h>
	
	int main(void){
		char nota;
		
		printf("====== PROGRAMA QUE MUESTRA UN MENSAJE DEPENDIENDO DE LA NOTA ======\n\n");
		printf("Ingrese la nota seg%cn el rango (A-F): ",163);
		scanf("%c", &nota);
		
		switch(nota){
			case 'A': case 'a':
				printf("Su nota es SOBRESALIENTE.\n");
				break;
			case 'B': case 'b':
				printf("Su nota es EXCELENTE.\n");
				break;
			case 'C': case 'c':
				printf("Su nota es MUY BUENO.\n");
				break;
			case 'D': case 'd':
				printf("Su nota es PASABLE.\n");
				break;
			case 'E': case 'e':
				printf("Su nota es MALA.\n");
				break;
			case 'F': case 'f':
				printf("Su nota es MUY MALA.\n");
				break;
			default:
				printf("La letra de la nota ingresada no es v%clida.\n", 160);
		}
	}
\end{lstlisting}


\section [Repetición while]{Repetición while}

Se utilizan para repetir comandos hasta que la condición sea falsa.

\vspace{1em}
Se lo conoce como ciclo condicionado.

\vspace{1em}
{Al ciclo while se lo conoce como pre chequeo, porque primero chequea la condición y después ejecuta los comandos.}

\vspace{1em}
En la condición del ciclo hay una o más variables involucradas.

\vspace{1em}
{El ciclo while puede que no se ejecute si de entrada la condición es falsa.}

\vspace{1em}
Las variables involucradas en la condición deben pasar por las siguientes etapas.

\begin{enumerate}
	\item \textbf{Inicialización}: Todas las variables tienen que haber sido inicializadas.
	\item \textbf{Prueba/condición}: Primero hace un chequeo del valor que está en la condición y después ejecuta los comandos.
	\item \textbf{Modificación}: Estar atento que al menos una variable de la condición debe de modificarse y se debe de asegurar que no entre en un ciclo infinito, tener mucho cuidado para no crear ciclos infinitos.
\end{enumerate}

\vspace{1em}
La sintaxis se muestra a continuación con una sentencia:

\begin{verbatim}
	while(condición)
		sentencia;
\end{verbatim}

\space{1em}
La sintaxis de varias sentencias se muestra a continuación:

\begin{verbatim}
	while(condición){
		sentencias1;
		sentencias1;
		...
	}
\end{verbatim}

\section [Ejercicio]{Ejercicio}
Solicitar al usuario el ingreso de varios números enteros, termina el programa cuando se hayan ingresado 5 números pares.

\begin{lstlisting}[language=C, caption={Ingreso de cinco números pares}, style=codigoenc]
	#include <stdio.h>
	
	int main(void){
		int numero;
		int contador = 0;
		int numerosParesIngresados = 0;
		
		printf("====== PROGRAMA QUE SOLO PERMITE INGRESAR CINCO NUMEROS PARES ======\n\n");
		
		while(numerosParesIngresados < 5){
			printf("Hay %d n%cmeros pares ingresados y se han ingresado en total %d n%cmeros, ingrese el siguiente n%cmero: ", numerosParesIngresados, 163, contador, 163, 163);
			scanf("%d", &numero);
			if((numero % 2) == 0){
				numerosParesIngresados++;
			}
			contador++;
			printf("\n");
		}
	}
\end{lstlisting}

\section [Repetición for]{Repetición for}

Se utilizan para repetir comandos hasta que la condición sea falsa y es también conocido como el ciclo incondicionado.

\vspace{1em}
Se lo utiliza cuando se requiere repetir comandos un número fijo de veces.

\vspace{1em}
La sintaxis se muestra a continuación:

\begin{verbatim}
	for(inicialización; condición/iteración; incremento)
	sentencias
\end{verbatim}

\space{1em}
Las partes se muestran a continuación:

\begin{itemize}
	\item \textbf{Inicialización}: Inicializa las variables de control del bucle, se pueden utilizar una o más variables de control.
	\item \textbf{Condición}: Contiene la expresión lógica donde hace que el ciclo realice las iteraciones mientras esta condición sea verdadera.
	\item \textbf{Incremento}: incrementa o decrementa las variables de control.
	\item \textbf{Sentencias}: Son las instrucciones que se repetirán.
\end{itemize}

\section [Ejercicio]{Ejercicio}
Solicitar al usuario N números enteros positivos y mostrar su raíz cuadrada.

\begin{lstlisting}[language=C, caption={Raíz cuadrana de N números positivos}, style=codigoenc]
	#include <stdio.h>
	
	int main(void){
		int cantidadDeRaices;
		int i;
		unsigned int valorDelNumero;
		
		printf("====== PROGRAMA QUE MUESTRA LA RAIZ CUADRADA DE LOS NUMEROS ENTEROS POSITIVOS ======\n\n");
		printf("Cu%cntos n%cmeros enteros positivos desea saber su ra%cz cuadrada: ", 160, 163, 161);
		scanf("%d", &cantidadDeRaices);
		
		for(i = 1; i <= cantidadDeRaices; i++){
			printf("\n");
			printf("Ingrese el valor #%d: ", i);
			scanf("%d", &valorDelNumero);
			printf("La ra%cz cuadrada de %d es: %.2f", 161, valorDelNumero, sqrt(valorDelNumero));
		}
	}
\end{lstlisting}












\section [Repetición do - while]{Repetición do - while}
{El ciclo do - while, se ejecuta al menos una vez.}

\vspace{1em}
Este ciclo primero se ejecuta y después chequea la condición.

\vspace{1em}
La sintaxis se muestra a continuación:

\begin{verbatim}
	do
		sentencias
	while(expresión);
\end{verbatim}

\space{1em}
Tener en cuenta que termina el comando con punto y coma.

\section [Ejercicio]{Ejercicio}
Dadas las edades de N personas se desea saber la edad promedio.

\begin{lstlisting}[language=C, caption={Raíz cuadrana de N números positivos}, style=codigoenc]
	#include <stdio.h>
	
	int main(void){
		int cantidadDePersonas;
		int i;
		int edad;
		int sumaDeEdades = 0;
		
		printf("====== PROGRAMA QUE MUESTRA EL PROMEDIO DE EDADES ======\n\n");
		
		printf("Cu%cntas edades va a ingresar: ", 160);
		scanf("%d", &cantidadDePersonas);
		
		for(i = 1; i <= cantidadDePersonas; i++){
			do{
				printf("El rango de edad es de 0 - 100, ingrese la edad # %d: ", i);
				scanf("%d", &edad);
			}while((edad < 0) || (edad > 100));
			sumaDeEdades += edad;
		}
		
		printf("El promedio de edad de las %d personas es: %d", cantidadDePersonas, sumaDeEdades / cantidadDePersonas);
	}
\end{lstlisting}