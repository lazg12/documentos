\chapter[Operadores]{Operadores}
\raggedright
Sirven para operar variables.

\section [Aritméticos]{Aritméticos}
Son para hacer operaciones aritméticas.
\vspace{1em}
\begin{itemize}
	\item \textbf{+}: Suma las variables de tipo entero y real, ejemplo: {$a + b;$}
	\item \textbf{-}: Resta las variables de tipo entero y real, ejemplo: {$a - b;$}
	\item \textbf{*}: Producto, multiplica las variables de tipo entero y real, ejemplo: {$a * b;$}
	\item \textbf{/}: Cociente, divide las variables de tipo entero y real, ejemplo: {$a / b;$}
	\item \textbf{\%}: Resto, se obtiene solo de la división en las variables de tipo entero, ejemplo:{$a \% b;$}
\end{itemize}
La prioridad de estos operadores es la siguiente:
\begin{enumerate}
	\item \textbf{+, -}: Como operador unitario, ejemplo: +25; Indica que 25 es positivo
	\item {\textbf{*, /}, \%: cuando se opera más de 2 variables}
	\item \textbf{+, -}: cuando se opera más de 2 variables
\end{enumerate}

\section [De dirección de memoria]{De dirección de memoria}
Sirven para trabajar con punteros.
\vspace{1em}
\begin{itemize}
	\item \textbf{\&}: Se obtiene la dirección de memoria, comando de ejemplo: {\&num};
\end{itemize}

\section [Relacionales]{Relacionales}
{Se utilizan para comparar elementos entre sí.}
\vspace{1em}
\begin{itemize}
	\item \textbf{==}: {Igual a, comando de ejemplo: {a == b;}}
	\item \textbf{!=}: {No es igual a, sentencia: {a != b;}}
	\item \textbf{>}: {Mayor que, comando de ejemplo: {a > b;}}
	\item \textbf{<}: {Menor que, sentencia: {a < b;}}
	\item \textbf{>=}: Mayor o igual que, comando de ejemplo: {a >= b;}
	\item \textbf{<=}: Menor o igual que, sentencia: {a <= b;}
\end{itemize}

\section [Lógicos]{Lógicos}
{Sirven para obtener un valor de verdad sea verdadero o falso dependiendo de sus cláusulas.}
\vspace{1em}
\begin{itemize}
	\item \textbf{!}: {Negación, comando de ejemplo: {!(a >= b);}}
	\item \textbf{\&\&}: {Y lógica, sentencia: {(a < b) \&\& (c > d);}}
	\item \textbf{||}: {O lógica, sentencia: {(a < b) || (c > d);}}
\end{itemize}
El siguiente orden es de mayor a menor prioridad: {!, \&\&, ||}

\section [A nivel de bits]{A nivel de bits}
{Sirven para operar a nivel de bits.}

\vspace{1em}
\begin{itemize}
	\item 
	\item {\Huge \textbf{\textit{\underline{¡¡¡CONSULTAR!!!}}}}
	\item 
\end{itemize}

\section [Asignación]{Asignación}
{Sirven para asignar valores a las variables.}

\vspace{1em}
\begin{itemize}
	\item \textbf{=}: {Asigna el valor de b a la variable a: (a = b);}
	\item \textbf{*=}: {Multiplica a por b y el valor lo asigna a la variable a: (a *= b);}
	\item \textbf{/=}: {Divide a por b y el valor lo asigna a la variable a: (a /= b);}
	\item \textbf{\%=}: {Calcula el resto de la división y el valor lo asigna a la variable a: (a \%= b);}
	\item \textbf{+=}: {Suma a por b y el valor lo asigna a la variable a: (a += b);}
	\item \textbf{-=}: {Resta a por b y el valor lo asigna a la variable a: (a-= b);}
\end{itemize}

\section [Prioridad]{Prioridad de operadores}
Convenio de qué se resuelve primero cuando hay varias operaciones.
\vspace{1em}
\begin{enumerate}
	\item \textbf{Matemáticos}
	\item \textbf{Relacionales}
	\item \textbf{Lógicos}
\end{enumerate}

\section [Incremento y decremento]{Incremento y decremento}
Sirven para incrementar o decrementar en una unidad la variable.
\vspace{1em}
\begin{itemize}
	\item {\textbf{a++}:} Pos incremento, primero asigna y después suma en una unidad.
	\item {\textbf{++a}:} Pre incremento, primero suma en una unidad y después asigna.
	\item {\textbf{a--}:} Pos decremento, primero asigna y después resta en una unidad.
	\item {\textbf{--a}:} Pre decremento, primero suma en una unidad y después asigna.
\end{itemize}