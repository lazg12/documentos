\chapter[Modularización]{Introducción a la modularización}
\raggedright
Módulo es cada una de las partes que resuelve un subproblema del problema.

\vspace{1em}
Cada módulo tiene una tarea bien definida.

\vspace{1em}
Los módulos para resolver su problema pueden ser usados por otros tipos de módulos.

\vspace{1em}
Un módulo se puede comunicar con otro a través de una interfaz asignada a la comunicación donde debe de estar bien definida.

\vspace{1em}
Existen dos tipos de módulos que son: funciones y procedimientos.

\vspace{1em}
La modularización utiliza aquella técnica llamada divide y vencerás.

\vspace{1em}
Para la resolución de un problema se descompone en módulos.

\vspace{1em}
Cada módulo se divide en nuevos módulos hasta que el problema se reduce a actividades básicas.

\vspace{1em}
Cuando se termine cada módulo se van fusionando y conectando los módulos hasta resolver el problema original.

\section[Ventajas]{Ventajas de modularizar código}

\begin{enumerate}
	\item Reutilización de código
	\item Ayuda al desarrollo y codificación ágil de los códigos en menos tiempo y poco esfuerzo
	\item Mejor modelización de los problemas
	\item Mayor legibilidad
	\item Etc.
\end{enumerate}