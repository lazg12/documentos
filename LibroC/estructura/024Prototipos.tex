\chapter[Prototipos]{Prototipos}
\raggedright
El prototipo es la declaración de una función o procedimiento, es su firma o también llamada cabecera.

\vspace{1em}
El prototipo no tiene cuerpo y terminan con punto y coma.

\vspace{1em}
{Los prototipos se encuentran antes del main, lo cual permiten al programa cómo debe de estar declarada y como utilizar la función o procedimiento donde la implementación se hace debajo del main.}

\section[Declaración]{Declaración de prototipos}
A continuación se muestra la sintaxis del prototipo:

\verb|tipo nombreDelPrototipo(Parámetros);|

\section[Parámetros formales]{Parámetros formales}
A las funciones y procedimientos los parámetros formales le da forma indicándole el tipo, orden y cantidad de parámetros que recibe.


\section[Parámetros actuales]{Parámetros actuales}
Son aquellos cuando se invocan a la función o procedimiento, donde son conocidos como argumentos porque tienen un valor determinado.

\vspace{1em}
Los parámetros actuales o argumentos deben de coincidir con los formales.
